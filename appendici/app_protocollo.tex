\chapter {Acquisizione dati e parametri pazienti}

La scelta dei pazienti da arruolare durante studi clinici può seguire due diverse logiche. La prima individua delle regole volte a minimizzare le differenze fra pazienti, sia per quanto riguarda gli aspetti antropometrici (età, peso, altezza, \ldots), sia per quelli relativi alla dialisi ($Q_{uf}$, età dialitica, tempo di dialisi, \ldots) in modo da ottenere un campione il più omogeneo possibile, al fine di indagare gli effetti di parametri specifici (e.g. pre- \textit{vs} post-diluizione). Con la seconda logica, detta \textit{random trial}, il campione da analizzare presenta caratteristiche meno omogenee al fine di individuare tutte le possibili cause relative a un particolare effetto. 
In questo lavoro sarebbe stato preferibile utilizzare la prima logica, per meglio individuare gli effetti dovuti alla modalità di diluizione.
In pratica, poiché il numero dei soggetti era limitato, non è stata effettuata alcuna selezione, e si sono rilevati i dati di tutti e solo i pazienti che hanno fornito il loro consenso informato per lo studio.
Qui di seguito sono elencati i dati raccolti.
\paragraph{Dati relativi al paziente}
\begin{itemize}
	\item dati anagrafici (nome, cognome, sesso, età);
	\item sito di prelievo (fistola, catetere venoso centrale);
	\item peso secco.
\end{itemize}

\paragraph{Dati della seduta dialitica}
\begin{itemize}
	\item Macchina dializzatrice (tutte Fresenius 5008S);
	\item codice del filtro, codice della sacca acida, sacca basica;
	\item tipologia HDF (pre- o post-diluizione);
	\item Volume dell'ultrafiltrato e tempo di dialisi impostato a inizio seduta;
	\item Ematocrito e proteinocrito impostati sulla macchina\footnote{non coincidenti con l'ematocrito e proteinocrito reali iniziali del paziente.}
	\item Portata del fluido di dialisi (impostata automaticamente dalla macchina);
	\item Portata di diluizione (impostata automaticamente dalla macchina);
	\item Peso iniziale del paziente;
	\item Tempistica dei prelievi ematici (inizio seduta, ogni ora, fine seduta);
	\item Annotazioni di anomalie e assunzione di cibo o bevande.
\end{itemize}

\paragraph{Registro infermieristico della seduta: dati rilevati con cadenza oraria}
\begin{itemize}
	\item peso del paziente;
	\item pressione sistolica e diastolica;
	\item frequenza cardiaca;
	\item pressione della linea venosa e arteriosa.
\end{itemize}

\noindent
Le sedute si sono svolte presso il centro dialisi San Faustino, situato in via S. Faustino 27, 20134 Milano (MI), avendo come referente il dr. Raffaele Galato. Qui sono stati monitorati, per quattro sedute consecutive, 6 pazienti sottoposti ad emodiafiltrazione, tutti in post-diluizione. I prelievi ematici sono stati effettuati dalla linea arteriosa in uscita dal paziente a inizio seduta, dopo ogni ora e a fine seduta. I campioni di sangue, raccolti in provette da $5$ $mL$, sono stati conservati in frigorifero fino alla fine della seduta dialitica e successivamente trasportati in un contenitore termo-isolante presso il laboratorio di analisi, per essere analizzati. Da questi campioni, dopo la centrifugazione a $3500$ $rpm$ per $15$ $min$, è stato prelevato $1$ $mL$ di plasma e conservato a $-80^{\circ}C$. I campioni sono conservati presso il LaBS (Laboratorio di meccanica delle Strutture Biologiche) del Politecnico di Milano.


%
%\begin{table}[!htb]\label{tab:dati}
%\centering\caption{Dati raccolti durante le sedute di HDF.}
%\advance\leftskip-50pt
%\begin{tabular}{lccccccccccc}
%\toprule
%paziente       & AMB     & BET     & EAS     & GAR & GRA & LI & ROD & SIM & VIN & ZAN \\
%\midrule
% sesso          & M       & F       & M       & F & M & M & M & M & F & F\\
% età            & 33      & 53      & 32      & 75 & 57 & 35 & 73 & 68 & 39 & 71\\
%sito prelievo   & fistola & fistola & fistola & fistola & fistola & fistola & fistola & fistola & fistola &  fistola\\
%peso secco (Kg) & 59,5    & 51      & 64,5    & 54 & 80 & 51 & 86 & 82 & 47 & 92\\
%tipo HDF        & pre     & post    & post    & post & post & post & post & post & pre & post\\
%Vuf (mL)        & 2800    & 1400    & 3200    & 2400 & 3200 & 3200 & 1600 & 1700 & 2800 & 3600\\
%Td (min)        & 240     & 210     & 240     & 210 & 240 & 240 & 210 & 210 & 210 & 240\\
%Quf (ml/h)      & 700     & 400     & 800     & 686 & 800 & 800 & 457 & 486 & 800 & 900\\
%Qb (ml/min)     & 250     & 300     & 260     & 300 & 300 & 300 & 300 & 300 & 200 & 300\\
%Qd (ml/min)     & 500     & 500     & 500     & 500 & 500 & 500 & 500 & 500 & 500 & 500\\
%Qs (ml/min)     & 135     & 86      & 67      & 90 & 72 & 80 & 84 & 84 & 104 & 54\\
%Hct\%           & 35      & 35      & 35      & 40 & 35 & 45 & 35 & 35 & 35 & 35\\
%Tp(gr/dL)       & 7,5     & 7,5     & 7,5     & 6,2 & 7,5 & 6,2& 7,5 & 7,5 & 7,5 & 7,5\\
%\bottomrule
%\end{tabular}
%\end{table}
