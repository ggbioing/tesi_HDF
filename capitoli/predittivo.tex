%èéÈàìòù   {     }
\documentclass[10pt,twoside]{book} 
\usepackage{amssymb,amsmath}         %pacchetto per la gestione delle formule matematiche
\usepackage[english, italian]{babel} % regole di sillabazione
\usepackage[ansinew]{inputenc}       %uso diretto di caratteri accentati
\usepackage[T1]{fontenc}
\usepackage[dvips]{epsfig,psfrag}    %pacchetto per la gestione delle figure *.eps
\usepackage{a4}                      % formato di stampa A4
\usepackage{cancel}     
\usepackage{booktabs}   

\usepackage{fancyhdr}
\usepackage{setspace}
\usepackage{graphicx}
\usepackage{subfigure}
\usepackage[hang,small,bf]{caption}
\setlength{\captionmargin}{20pt}
\usepackage{a4}
\usepackage{natbib}
\usepackage{indentfirst}
\usepackage{wrapfig}
\usepackage{tabularx}

\begin{document}
\chapter{Risultati analisi predittive}
 Il presente capitolo si propone di valutare, tramite analisi dati, quanto i parametri caratteristici del modello di una determinata seduta possano risultare validi al fine di prevedere gli andamenti delle concentrazioni plasmatiche delle sedute successive.
 Si effettueranno infatti dei confronti tra casistiche descrittive \footnote{il termine descrittivo si riferisce alla sedute i cui parametri sono quelli ottenuti a seguito dell'analisi di sensitività e ottimizzazione}, ovvero le sedute cosiddette gold standard, e casistiche predittive\footnote{il termine predittivo si riferisce invece all'utilizzo dei parametri ottimizzati di una o più sedute precedenti a quella presa in analisi}.
 \section{Predizione tramite parametri di una sola seduta}
\begin{table}[htb]
	\centering
	\caption{Capacità descrittiva del modello: errori di stima $e_i$ ed errori assoluti di stima $|e_i|$. Media ($\mu$) e deviazione standard ($\sigma$). Numerosità del campione $N=87$.}\label{tab:descrizione}
	\begin{tabular}{lrrrrrrrr}
	\toprule 
		\textbf{Soluto}   &  \multicolumn{8}{c}{\textbf{Errori di simulazione (\%)}}  \\
		\cmidrule(lr){2-9}
				              &        \multicolumn{4}{c}{$e_i$}             &       \multicolumn{4}{c}{$|e_i|$}             \\
		                  & \multicolumn{1}{c}{$\mu$}      & \multicolumn{1}{c}{$\sigma$}   & $min$   & $max$   & \multicolumn{1}{c}{$\mu$}     & \multicolumn{1}{c}{$\sigma$}   & $min$   & $max$  \\
    \cmidrule(lr){2-5}\cmidrule(lr){6-9}
    %                    media       dev.std.        min        max    media ass   dev.st.        min       max
  	Sodio             & $ 0,10$     & $ 0,92$    & $ -1,93$  & $2,12$  & $0,78$   & $0,48$     & $0,01$ & $ 2,12$   \\
  	Potassio          & $-0,53$     & $ 4,49$    & $-11,86$  & $7,57$  & $3,31$   & $3,05$     & $0,02$ & $11,86$  \\
  	Cloro             & $-0,02$     & $ 0,60$    & $ -1,32$  & $1,82$  & $0,46$   & $0,38$     & $0,01$ & $ 1,82$   \\
  	Calcio            & $ 0,25$     & $ 2,15$    & $ -5,18$  & $10,62$ & $1,48$   & $1,57$     & $0,03$ & $10,62$  \\
  	Fosfato           & $ 1,44$     & $ 9,13$    & $-15,93$  & $32,98$ & $6,69$   & $6,32$     & $0,44$ & $32,98$  \\
  	Magnesio          & $ 3,75$     & $ 5,6 $    & $ -7,60$  & $18,23$ & $5,28$   & $4,16$     & $0,05$ & $18,23$  \\
  	Urea              & $-3,19$     & $10,95$    & $-23,34$  & $18,83$ & $9,13$   & $6,74$     & $0,35$ & $23,34$  \\
  	Creatinina        & $-4,03$     & $10,43$    & $-26,67$  & $16,42$ & $9,21$   & $6,27$     & $0,06$ & $26,67$  \\
  	Proteine          & $ 7,33$     & $ 4,94$    & $ -3,32$  & $18,55$ & $7,51$   & $4,66$     & $0,16$ & $18,55$  \\
  \bottomrule
\end{tabular}
\end{table}

\begin{table}[htb]
	\centering
	\caption{Capacità predittiva del modello: errori di stima $e_i$ ed errori assoluti di stima $|e_i|$. Media ($\mu$) e deviazione standard ($\sigma$). Numerosità del campione $N=87$.}\label{tab:descrizione}
	\begin{tabular}{lrrrrrrrr}
	\toprule 
		\textbf{Soluto}   &  \multicolumn{8}{c}{\textbf{Errori di simulazione (\%)}}  \\
		\cmidrule(lr){2-9}
				              &        \multicolumn{4}{c}{$e_i$}             &       \multicolumn{4}{c}{$|e_i|$}             \\
		                  & \multicolumn{1}{c}{$\mu$}      & \multicolumn{1}{c}{$\sigma$}   & $min$   & $max$   & \multicolumn{1}{c}{$\mu$}     & \multicolumn{1}{c}{$\sigma$}   & $min$   & $max$  \\
    \cmidrule(lr){2-5}\cmidrule(lr){6-9}
    %                    media       dev.std.        min        max      media ass   dev.st.        min       max
  	Sodio             & $ -2,25$    & $ 5,82$    & $ -12,93$  & $9,01$   & $5,05$   & $3,61$     & $0,00$ & $12,93$   \\
  	Potassio          & $-2,79$     & $ 8,06$    & $-21,73$   & $15,58$  & $6,97$   & $4,85$     & $0,04$ & $21,73$  \\
  	Cloro             & $-1,47$     & $ 4,14$    & $ -8,45$   & $6,40$   & $3,74$   & $2,27$     & $0,12$ & $ 8,45$   \\
  	Calcio            & $-9,11$     & $ 15,11$   & $ -35,38$  & $21,72$  & $14,25$  & $10,31$    & $0,20$ & $35,38$  \\
  	Fosfato           & $-4,93$     & $ 15,99$   & $-37,27$   & $27,73$  & $12,41$  & $11,14$    & $0,25$ & $37,27$  \\
  	Magnesio          & $ 4,00$     & $ 6,45$    & $ -8,65$   & $14,9$   & $6,44$   & $3,96$     & $0,17$ & $14,90$  \\
  	Urea              & $-9,91$     & $13,99$    & $-34,92$   & $20,87$  & $14,41$  & $9,18$     & $0,12$ & $34,92$  \\
  	Creatinina        & $-9,71$     & $12,09$    & $-35,08$   & $16,39$  & $13,01$  & $8,37$     & $0,81$ & $35,08$  \\
  	Proteine          & $ 7,62$     & $5,07$     & $ -4,27$   & $18,48$  & $7,96$   & $4,51$     & $0,01$ & $18,48$  \\
  \bottomrule
\end{tabular}
\end{table}
Le due tabelle contengono i valori degli scostamenti medi, varianze e valori massimi e minimi. La prima è stata ottenuta tramite l'analisi dei valori relativi alle sedute due, tre e quattro, mentre la seconda considera i valori degli scostamenti delle uscite delle sedute due tre e quattro quando queste vengono implementate con i valori di $\rho$, $\eta$ e k della prima seduta.
Quello che si nota è che la media degli scostamenti dei valori assoluti per le sedute gold standard risulta inferiore al $10\%$, mentre per le sedute successive con i valori dei paramteri della prima, la media si alza raggiungendo valori anche attorno al $14\%$. Stesso andamento per quanto riguarda le deviazioni standard: passando da caso descrittivo a caso predittivo i valori aumentano. L'unica variazione non significativi si ha a livello delle proteine, probabilmente \dots.
\begin{tabular}{lcccr}
\toprule
Soluto & $0-5\%$ & $5-10\%$ & $10-20\%$ & $>20\%$\\
\midrule
$1->2$& 43,94 & 21,72 & 21,21 & 13,13\\
$1->3$ & 36,36 & 26,77 & 24,75 & 12,12\\ 
$1->4$& 34,85 & 25,25 & 28,79 & 11,11\\
$2->2$& 64,14 & 17,68 & 17,17 & 1,01\\
$3->3$ &63,49 &  20,63 & 15,87 & 0,00\\
$4->4$& 66,16 &  15,66 & 15,15 & 3,03\\
\bottomrule
\caption{Percentuali scostamenti predittivi e descrittivi }
\end{tabular}


\section{Predizione tramite parametri delle due sedute precedenti}

\begin{table}[htb]
	\centering
	\caption{Capacità descrittiva del modello: errori di stima $e_i$ ed errori assoluti di stima $|e_i|$. Media ($\mu$) e deviazione standard ($\sigma$). Numerosità del campione $N=87$.}\label{tab:descrizione}
	\begin{tabular}{lrrrrrrrr}
	\toprule 
		\textbf{Soluto}   &  \multicolumn{8}{c}{\textbf{Errori di simulazione (\%)}}  \\
		\cmidrule(lr){2-9}
				              &        \multicolumn{4}{c}{$e_i$}             &       \multicolumn{4}{c}{$|e_i|$}             \\
		                  & \multicolumn{1}{c}{$\mu$}      & \multicolumn{1}{c}{$\sigma$}   & $min$   & $max$   & \multicolumn{1}{c}{$\mu$}     & \multicolumn{1}{c}{$\sigma$}   & $min$   & $max$  \\
    \cmidrule(lr){2-5}\cmidrule(lr){6-9}
    %                    media       dev.std.        min        max    media ass   dev.st.        min       max
  	Sodio             & $ 0,10$     & $ 0,94$    & $ -1,93$  & $2,12$  & $0,78$   & $0,52$     & $0,01$ & $ 2,12$   \\
  	Potassio          & $-0,55$     & $ 4,34$    & $-11,86$  & $7,03$  & $3,17$   & $2,98$     & $0,24$ & $11,86$  \\
  	Cloro             & $-0,03$     & $ 0,52$    & $ -0,95$  & $1,62$  & $0,39$   & $0,33$     & $0,01$ & $ 1,62$   \\
  	Calcio            & $ 0,30$     & $ 2,50$    & $ -5,18$  & $10,62$ & $1,70$   & $1,85$     & $0,03$ & $10,62$  \\
  	Fosfato           & $ 1,54$     & $ 9,10$    & $-11,89$  & $32,98$ & $6,25$   & $6,73$     & $0,44$ & $32,98$  \\
  	Magnesio          & $ 3,83$     & $ 6,01$    & $ -7,60$  & $18,23$ & $5,33$   & $4,69$     & $0,05$ & $18,23$  \\
  	Urea              & $-3,01$     & $10,64$    & $-23,34$  & $18,83$ & $8,78$   & $6,60$     & $0,44$ & $23,34$  \\
  	Creatinina        & $-3,74$     & $10,12$    & $-21,80$  & $16,42$ & $8,88$   & $6,00$     & $0,51$ & $21,80$  \\
  	Proteine          & $ 8,07$     & $ 4,85$    & $ -3,32$  & $18,55$ & $8,24$   & $4,56$     & $0,16$ & $18,55$  \\
  \bottomrule
\end{tabular}
\end{table}

\begin{table}[htb]
	\centering
	\caption{Capacità predittiva del modello: errori di stima $e_i$ ed errori assoluti di stima $|e_i|$. Media ($\mu$) e deviazione standard ($\sigma$). Numerosità del campione $N=87$.}\label{tab:descrizione}
	\begin{tabular}{lrrrrrrrr}
	\toprule 
		\textbf{Soluto}   &  \multicolumn{8}{c}{\textbf{Errori di simulazione (\%)}}  \\
		\cmidrule(lr){2-9}
				              &        \multicolumn{4}{c}{$e_i$}             &       \multicolumn{4}{c}{$|e_i|$}             \\
		                  & \multicolumn{1}{c}{$\mu$}      & \multicolumn{1}{c}{$\sigma$}   & $min$   & $max$   & \multicolumn{1}{c}{$\mu$}     & \multicolumn{1}{c}{$\sigma$}   & $min$   & $max$  \\
    \cmidrule(lr){2-5}\cmidrule(lr){6-9}
    %                    media       dev.std.        min        max      media ass   dev.st.        min       max
  	Sodio             & $ -1,10$    & $ 5,75$    & $ -10,55$  & $11,40$   & $4,60$   & $3,55$     & $0,00$ & $11,40$   \\
  	Potassio          & $-3,57$     & $ 10,32$    & $-9,00$   & $31,82$  & $7,63$   & $7,74$     & $0,20$ & $31,82$  \\
  	Cloro             & $-0,20$     & $ 4,14$    & $ -7,16$   & $9,55$   & $3,03$   & $2,79$     & $0,03$ & $ 9,55$   \\
  	Calcio            & $1,54$     & $ 20,30$   & $ -28,65$  & $51,90$  & $14,39$  & $14,23$    & $0,29$ & $51,90$  \\
  	Fosfato           & $3,49$     & $ 20,66$   & $-36,14$   & $42,88$  & $16,16$  & $13,11$    & $0,05$ & $42,88$  \\
  	Magnesio          & $ 4,82$     & $ 6,88$    & $ -8,65$   & $14,77$   & $7,24$   & $4,17$     & $0,26$ & $14,77$  \\
  	Urea              & $-4,59$     & $15,33$    & $-27,49$   & $34,61$  & $13,81$  & $7,83$     & $0,37$ & $34,61$  \\
  	Creatinina        & $-3,82$     & $13,33$    & $-25,73$   & $26,20$  & $12,12$  & $6,49$     & $0,71$ & $26,20$  \\
  	Proteine          & $ 8,61$     & $4,45$     & $ -3,06$   & $18,71$  & $8,75$   & $4,15$     & $0,84$ & $18,71$  \\
  \bottomrule
\end{tabular}
\end{table}

In questo caso i dati utilizzati per predire le sedute tre e quattro sono quelli delle sedute uno e due. Per il calcolo egli scostamenti delle sedute tre e quattro è stata effettuata la media dei parametri delle sedute uno e due e inseria poi nelle seute tre e quattro per valutare le uscite.
\section{Stessa seduta predetta con uno, due e tre parametri} 
Per l'analisi della predittività è stata considerata una sola seduta, più precisamente la quarta, i cui dati iniziali sono stai inseriti nel modello ma variando i parametri $rho$ $eta$ e $k$, presi rispettivamente dalla prima seduta, dalla media della prima e della seconda seduta e dalla media delle tre sedute precedenti. I risultati sono mostrati in tabella.



\end{document}