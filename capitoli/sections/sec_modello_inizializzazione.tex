\section{Inizializzazione del sistema}
Le equazioni differenziali che descrivono la dinamica dei soluti durante l'emodiafiltrazione, illustrate singolarmente nei precedenti paragrafi, sono raggruppate nel seguente sistema:
\begin{align}
		&\dot{M}_{ic}^{(s)} = \Phi_{ic}^{(s)}                                    \label{dMic}\\
		&\dot{M}_{ex}^{(s)} = -\Phi_{ic}^{(s)} -\Phi_{hdf}^{(s)} + Q_s C_s^{(s)} \label{dMex}\\ 
		&\dot{V}_{ic}       = Q_{ic}                                             \label{dVic}\\
		&\dot{V}_{is}       = - Q_{ic} + Q_{fc}                                  \label{dVis}\\
		&\dot{V}_{pl}       = -Q_{fc} -Q_{uf}                                    \label{dVpl}\\
		&\dot{P}_{ac}       = \frac{1}{C_c} \dot{V}_{pl}                         \label{dPac}\\
		&\dot{P}_{is}       = E_{is}\dot{V}_{is}                                 \label{dPis}
\end{align}
Questo è un sistema di $2N+5$ equazioni differenziali, in cui $N$ è il numero di soluti di cui si vuole descrivere la dinamica\footnote{in questa trattazione $N=9$, e i soluti sono: sodio, potassio, cloro, calcio, fosfato inorganico, magnesio, urea, glucosio e proteine.}. Trattandosi di un sistema del primo ordine, affinché sia risolvibile è necessario fornire i valori iniziali delle variabili.

Per quanto riguarda i volumi, in Guyton et al. \cite{guyton} si afferma che in un uomo adulto che pesa $70$ $kg$ l'acqua totale presente nell'organismo ammonta a circa il 60\% del peso corporeo\footnote{questa percentuale puo variare a seconda dell'età, del sesso e del grado di obesità.}, ed è distribuita nei tre compartimenti principali come indicato in \tablename~\ref{tab:volumi_guyton}.
\begin{table}[htb]
	\centering
	\caption{Distribuzione dei fluidi corporei (fonte: Guyton et al. \cite{guyton})}
	\begin{tabular}{lcccc}
	\toprule 
		                           & \textbf{Plasma}  & \textbf{Interstizio} & \textbf{Cellule} & \textbf{Totale} \\
  \midrule
  	\textbf{Volume} (Litri)        &  $3$                 &  $11$                  &  $28$                   & $42$            \\
   \bottomrule
\end{tabular}\label{tab:volumi_guyton}
\end{table}
A partire da queste informazioni, se si indica con $W_0$ il peso a inizio dialisi del paziente, è possibile inizializzare i volumi dei tre compartimenti con le seguenti equazioni:
\begin{align*}
		V_{tot,0}      &= 0,6\cdot W_0          \\
		V_{pl,0}       &= 3/42  \cdot V_{tot,0} \\
		V_{is,0}       &= 11/42 \cdot V_{tot,0} \\
		V_{ic,0}       &= 28/42 \cdot V_{tot,0} 
\end{align*}

Proseguendo con l'inizializzazione delle variabili, è possibile combinare le equazioni (\ref{eq:ovvia}) e (\ref{eq:meno_ovvia}) con i calcoli accessori esposti in appendice~\ref{app:A} e ricavare il seguente algoritmo di inizializzazione delle masse molari dei soluti:
\begin{align*}
		C_{pl,0} &= \text{\emph{dati clinici noti}} \\
		C_{is,0} &= \alpha_d \cdot C_{pl,0}  \\
    C_{ic,0} &= \beta \cdot C_{is,0}     \\
    M_{ic,0} &= C_{ic,0} \cdot Vic0      \\
    M_{ex,0} &= C_{is,0} \cdot (V_{pl,0}/\alpha_d+ V_{is,0})
\end{align*}

Le pressioni ai capillari arteriosi e nell'interstizio sono inizializzate come in Ursino et al. \cite{ursino}, cioè:
\begin{align*}
		P_{ac,0} &= 35 \text{ \emph{mmHg}} \\
		P_{is,0} &= -5,97 \text{ \emph{mmHg}}
\end{align*}