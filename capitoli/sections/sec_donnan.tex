\section{Effetto Donnan}\label{sec:donnan}
Quando una membrana è permeabile a uno o più soluti, due soluzioni a concentrazioni diverse separate da tale membrana tenderanno, per il fenomeno della diffusione, a raggiungere lo stato stazionario in cui il gradiente di concentrazione di ogni soluto si annulla. Tuttavia, in un sistema in cui due soluzioni contenenti elettroliti sono separate da una membrana permeabile alla maggior parte degli ioni, ma \textit{impermeabile ad almeno uno di essi}, si nota che le concentrazioni degli ioni permeanti $(s)$, all'equilibrio, hanno gradiente non nullo ai capi della membrana, cioè:
\begin{equation}\label{donnan}
	C_{2,eq.}^{(s)} = \alpha \cdot C_{1,eq.}^{(s)}
\end{equation}
\noindent
\newline
Questo fenomeno è dovuto all'effetto \textit{Donnan} \cite{articolo_donnan} e $\alpha$ è chiamato \textit{coefficiente di Donnan}. $C_1$ è il compartimento in cui è presente la specie non diffondibile attraverso la membrana. Nella trattazione di seguito presentata la causa di questo effetto sono le proteine.

L'effetto Donnan, chiamato anche di \textit{Gibson-Donnan}, è basato sul mantenimento dell'elettroneutralità nei due compartimenti separati dalla membrana. Si prenda come esempio la membrana capillare, che separa il compartimento sangue dall'interstizio o, analogamente, la membrana del dializzatore che separa il sangue dal liquido dializzante: si tratta di membrane praticamente impermeabili alle proteine. Nel compartimento sangue le proteine, cariche negativamente, tenderanno per interazione elettrostatica a trattenere i cationi e a respingere gli anioni, ostacolandone e facilitandone rispettivamente la diffusione. Di conseguenza, il fattore di Donnan $\alpha$, definito per ogni soluto dall'equazione~\ref{donnan}, sarà minore di uno per i cationi e maggiore di uno per gli anioni \cite{ursino}.

