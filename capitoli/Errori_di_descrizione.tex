%èéÈàìòù   {     }
\documentclass[10pt,twoside]{book} 
\usepackage{amssymb,amsmath}         %pacchetto per la gestione delle formule matematiche
\usepackage[english, italian]{babel} % regole di sillabazione
\usepackage[ansinew]{inputenc}       %uso diretto di caratteri accentati
\usepackage[T1]{fontenc}
\usepackage[dvips]{epsfig,psfrag}    %pacchetto per la gestione delle figure *.eps
\usepackage{a4}                      % formato di stampa A4
\usepackage{cancel}     
\usepackage{booktabs}   

\usepackage{fancyhdr}
\usepackage{setspace}
\usepackage{graphicx}
\usepackage{subfigure}
\usepackage[hang,small,bf]{caption}
\setlength{\captionmargin}{20pt}
\usepackage{a4}
\usepackage{natbib}
\usepackage{indentfirst}
\usepackage{wrapfig}

\begin{document}
\chapter{Risultati}
Qui di seguito verranno riportati ed analizzati i dati ottenuti in uscita dal modello.
Il modello fornisce i dati relativi alle concentrazioni plasmatiche (interstiziali e intracellulari) di sodio, potassio, cloro, calcio, fosforo, magnesio, urea, cratina e proteine durante una seduta di emodiafiltrazione. Per ogni uscita del modello sono stati calcolati gli scostamenti rispetto ai valori ottenuti tramite prelievi ematici ad ogni ora da inizio a fine seduta. I prelievi ematici sono stati effettuati per quattro sedute successive su 6 pazienti. Per errori di descrizione si intendono gli errori commessi dal modello su una determinata seduta utilizzando i parametri della seduta specifica
I valori riportati in tabella sono stati ricavati considerando i valori di tutte e quattro le sedute effettuate sui sei pazienti .

\begin{tabular}{lcccr}
\toprule
Soluto & $\eta$ & $\sigma$ & min & max\\
\midrule
$Na^{++}$& 0,105 & 0,931 & -1,928 & 2,696\\
$k^{+}$ & -0,509 & 4,444 & -11,856 & 9,425\\ 
$Cl^{-}$& -0,024 & 0,585 & -1,323 & 1,822\\
$Ca^{++}$& 0,202 &  1,914 & -5,183 & 10,622\\
$Ph$ & 1,919 &  9,416 & -15,925 & 32,976\\
$Mg^{++}$& 3,103 &  5,351 & -7,939 & 18,232\\
Urea & -3,13 & 10,532 & -23,341 & 18,828\\
Glucosio & -3,786 & 10,024 & -26,67 & 16,422\\
Proteine & 7,097 &  4,647 & -3,323 & 18,546\\
\bottomrule
\caption{Errori di Descrizione }
\end{tabular}

\begin{tabular}{lcccr}
\toprule
Soluto & $\eta$ & $\sigma$ & min & max\\
\midrule
$Na^{++}$& 0,7821 & 0,510 & 0,007 & 2,696\\
$k^{+}$ & 3,3326 & 2,963 & 0,024 & 11,856\\ 
$Cl^{-}$ & 0,4531 & 0,368 & 0,009 & 1,822\\
$Ca^{++}$ & 1,288 &  1,424 & 0,029 & 10,622\\
Ph & 6,932 &  6,616 & 0,048 & 32,976\\
$Mg^{++}$& 4,795 &  3,887 & 0,046 & 18,232\\
Urea & 8,884 & 6,403 & 0,35 & 23,241\\
Glucosio & 8,826 & 6,016 & 0,05 & 26,673\\
Proteine & 7,230 &  4,435 & 0,162 & 18,546\\
\bottomrule
\caption{Errori di Descrizione Assoluti}
\end{tabular}


\end{document}