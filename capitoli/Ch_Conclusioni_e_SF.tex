\chapter{Conclusioni e sviluppi futuri}\label{ch:end}
Lo scopo principale di questa tesi è stato quello di sviluppare e implementare un modello matematico che permetta la simulazione di una procedura dialitica chiamata \textit{emodiafiltrazione on-line}. Ad oggi, nella letteratura di pertinenza, non si trovano modelli che permettano di progettare a priori tale tipo di trattamento. Per questo lavoro ci si è basati su modelli precedenti, sviluppati per la simulazione di sedute di emodialisi (\textsection~\ref{sec:stato}), le cui equazioni sono state riviste e adattate per descrivere la nuova tecnica dialitica (\chaptername~\ref{ch:online}).
Dopo aver constatato alcune limitazioni del modello sviluppato nel descrivere e predirre la dinamica dei soluti di interesse (Capitoli \ref{ch:descrizione} e \ref{ch:predizione}), si sono identificati dei possibili spazi d'azione per delle modifiche strutturali volte al miglioramento del modello stesso (introduzione dei fenomeni di adsorbimento per le proteine e dinamiche multicompartimentali per calcio, fosfato e magnesio).

I dati relativi alle sedute dialitiche sono stati registrati manualmente, in base alle informazioni lette sul monitor della macchina di dialisi ad inizio seduta e, periodicamente, in concomitanza dei prelievi ematici orari. Da queste letture si è verificata la tendenza della macchina a mantenere invariati i parametri di dialisi (es. $Q_{uf}$), tranne nei casi in cui si verificano otturazioni degli accessi vascolari. La possibilità di reperire questi dati in maniera automatica dalla dializzatrice permetterebbe di utilizzare questo flusso informativo come ingresso diretto al modello, aumentandone la precisione ed eliminando il problema di possibili errori umani di immissione dati. Si potrebbe anche usare il contenuto informativo delle pressioni di transmembrana del dializzatore per pilotare, nel modello, il fenomeno di trasporto convettivo.

Attualmente la dinamica di ogni soluto è regolata attraverso due parametri specifici (il coefficiente di trasferimento di massa attraverso la membrana cellulare $k$ e il coefficiente della capacità estrattiva del filtro dializzatore $\eta$) più un terzo parametro di carattere generale (coefficiente di permeabilità capillare $\rho$). Un ampio spazio per l'individuazione di possibili modifiche migliorative è fornito dall'analisi di sensitività di questi parametri sulle uscite del modello (Capitolo~\ref{ch:ottimizzazione}).

Successivamente alla riduzione degli errori di simulazione, si è pensato di indagare quali differenze ci siano fra la pre- e la post-diluizione e a cosa siano dovute, individuando nel grado di compartimentazione e nella carica elettrica dei soluti due possibili elementi che, insieme alla modalità di diluizione, ne influenzano l'andamento. A questo risultato si è giunti per via teorica, e fra gli sviluppi futuri ci sarebbe sicuramente da includere una conferma sperimentale di quanto teorizzato.
