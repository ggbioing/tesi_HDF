\chapter*{Summary}
Dialysis is a treatment allowing to purify the blood from uremic toxins. Dialysis also induce a complex fluid dynamic and substances exchange between the various human body compartments: from the plasma compartment, which is the most easely accessible, to the intracellular one, deeper and inaccessible. It is important for therapeutic purposes, to be able to describe and predict these dynamics, so that we can better tailor the therapy on every patient, by choosing among the technical options available nowadays. In fact, with respect to the recent past when the main treatment was the \textit{hemodialysis}, now is possible to choose several alternatives, among which we can find the \textit{hemodiafiltration} (HDF).
The use of mathematical modeling can help in the choice of the most suitable dialysis, and this thesis aims to provide such a tool for the special case of \textit{on-line} HDF.

The model here developed is based on the kinetic description of some blood solutes, considered to be important to monitor (e.g. sodium, potassium, chloride, calcium, inorganic phosphate, magnesium, urea, creatinine, proteins); is based on the fluids exchange between the different compartments, and on pressure changes, during a dialysis session performed with the HDF technique. In particular, we have tried to mathematically synthesized in a single system of ordinary differential equations, the interactions that occur between the following three systems:
\begin{enumerate}
	\item the ``patient'' system , described as three-compartmental (plasma, interstitial and intracellular) separated by semipermeable membranes;
	\item the ``online'' system for the production of the dialysate and replacement fluids;
	\item the  dialyzing filter, which is the site were mass and volume exchange take place.
\end{enumerate}
In the ``patient'' system, the mass exchange is two-compartmental (the intracellular compartment and the extracellular one, the latter formed by the plasma compartment plus the interstitial one). The fluids exchange is modeled as three-compartmental (plasma, interstitial cells).

The model, implemented in the software for the numerical calculation Matlab\textsuperscript\textregistered{} 7 (R14), requires the input of initial data from the patient and from the dialysis machine.

Three parameters were introduced and analyzed: the mass transfer coefficient to the cell membrane $k$, the dimensionless parameter $\rho$, accounting for the variability  of permeability in capillary walls, and the coefficient $\eta$, defined as the dialyzer extraction capability, which is dependent on the thickness of protein layer formed along the filter capillaries throuout the dialysis. Sensitivity analysis was also performed in order to understand which of this parameters affect most the model output. In order to assess sensitivity, parameters were individually modified by $\pm 10\%$ of their reference value, while keeping constant all the others. Variations in the output were recorded. Results from this kind of analysis are that the parameter for a given solute influences both, predictably, the dynamics of the solute itself, but also that of other solutes, thus highlighting the fact that the model has been developed with solutes coexisting and interacting in the same environment.

One of the ultimate goals was to evaluate  for how long the parameters identified in the first session would be valid for the prediction of the following ones, so that we can provide a useful tool for medical simulation. The model presented here shows overall results in line and close to those obtained by clinical observations and can therefore be considered, with few exceptions, a valid tool for prediction. In fact, deviations from real plasma concentrations of almost all solutes considered here, are at most $20\%$. In particular, the best results were obtained for sodium, potassium, chlorine, calcium and magnesium, with deviation of at most $10\%$. If we consider the average values, all the descriptive errors falls below $10\%$. Detailed analysis for every solute is given in the last chapters, along with some comments on how to improve the model descriptive capabilities.

Eventually, it is shown how the model can be used to detect the differences between the available dilution techniques (pre, post) during on-line HDF, by evaluating their different effects on the solutes dynamic throughout the dialysis. The degree of compartmentalization and the electric charge of the solutes has been identified as factors determining the extraction capabilities of both dilution techniques available for on-line HDF.