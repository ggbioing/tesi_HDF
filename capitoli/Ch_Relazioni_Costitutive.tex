\chapter{Relazioni Costitutive}\label{ch:rc}

\begin{flushright}
\textit{``Finché le leggi della matematica si riferiscono alla\\ realtà, non sono certe, e finché sono certe\\ non si riferiscono alla realtà.''}
\end{flushright}
\begin{flushright} \textsc{Albert Einstein} (1879 - 1955) \end{flushright} 
\begin{flushright} { } \end{flushright}

Col termine \textit{relazioni costitutive} si intendono quelle relazioni matematiche sperimentali che, attingendo ai concetti della fisica classica, sevono per descrivere macroscopicamente e in maniera schematica i fenomeni naturali. In questo capitolo saranno trattate tutte le relazioni utili alla costruzione di un modello per la descrizione dell'interazione fra organismo umano e macchina dializzatrice. 


\input capitoli/sections/sec_ProGenSis.tex
\input capitoli/sections/sec_membrane.tex
\input capitoli/sections/sec_diffusione.tex
\input capitoli/sections/sec_donnan.tex
\input capitoli/sections/sec_trasporto_convettivo.tex
\input capitoli/sections/sec_osmosi.tex
\input capitoli/sections/sec_pressione_oncotica.tex
\input capitoli/sections/sec_trasp_cellulare.tex
\input capitoli/sections/sec_dializzatore.tex



%\input capitoli/sections/sec_conclusioni_RC.tex
