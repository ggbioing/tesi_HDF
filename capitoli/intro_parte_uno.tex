\chapter*{Struttura della tesi}
Il presente lavoro è composto da nove capitoli e quattro appendici, raggruppati in quattro parti principali: 
\begin{itemize}
\item \textbf{Parte I:} Introduzione;
\item \textbf{Parte II:} Materiali e Metodi;
\item \textbf{Parte III:} Risultati e Discussioni;
\item \textbf{Parte IV:} Appendici e Bibliografia.
\end{itemize}
\paragraph{Parte I: Introduzione.}
La parte introduttiva (Cap.~1) è dedicata ad una breve descrizione della fisiologia dei reni umani e delle principali patologie che li colpiscono, seguita da un'esposizione delle possibili tecniche dialitiche per la sostituzione della funzionalità renale. Il Cap.~2 è interamente dedicato alla descrizione dell'on-line HDF, tecnica oggetto di modellizzazione in questa tesi.

\paragraph*{Parte II: Materiali e Metodi.}
In questa parte, dopo il breve Cap.~\ref{ch:math} di presentazione della modellistica matematica in generale, si passa, nel Cap.~\ref{ch:rc}, all'enunciazione delle principali leggi empiriche, dette relazioni costitutive, costituenti i pezzi fondamentali che compongono il modello finale, descritto successivamente nel Cap.~\ref{ch:online}.
Nel Cap.~\ref{ch:ottimizzazione} è descritta la procedura seguita per l'ottimizzazione dei parametri usati per la simulazione numerica, procedura basata sull'analisi della matrice di sensitività relativa al modello.

\paragraph*{Parte III: Risultati e Discussioni.}
In questa parte sono analizzate le proprietà descrittive (Cap.~\ref{ch:descrizione}) e predittive (Cap.~\ref{ch:predizione}) del modello. Relativamente alle concentrazioni plasmatiche dei soluti di interesse, gli scostamenti percentuali relativi fra simulazione numerica e realtà clinica sono mediamente inferiori al $20\%$. Sugli elettroliti gli errori di predizione medi sono ancora più bassi e inferiori al $10\%$. Questa parte si conclude con delle considerazioni finali e l'indicazione di alcuni possibili sviluppi futuri (Cap.~\ref{ch:end}).

\paragraph*{Parte IV: Appendici e Bibliografia.}
Nell'Appendice A e B sono indicati alcuni calcoli accessori utili per la modellistica lato paziente e lato dializzatore.
Nell'appendice C si descrive il protocollo di acquisizione dei dati raccolti presso il centro dialisi di riferimento.
Nell'appendice D si riporta infine il listato completo del codice usato per l'implementazione del modello.