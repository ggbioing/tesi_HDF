\chapter*{Sommario}
Le procedure dialitiche di depurazione del sangue dalle tossine uremiche, inducono un complesso scambio di fluidi e di sostanze fra i vari compartimenti dell'organismo umano: da quello plasmatico, più facilmente accessibile, a quello intracellulare, più profondo e inaccessibile. È importante, a fini terapeutici, essere in grado di descrivere e predire queste dinamiche, così che si possa personalizzare al meglio la terapia in base anche alle esigenze di ogni singolo paziente e in base alle opzioni tecniche oggi disponibili. Infatti, rispetto al recente passato in cui la principale terapia di depurazione era quella dell'\textit{emodialisi}, oggi è possibile scegliere fra più alternative, fra le quali vi è l'\textit{emodiafiltrazione} (HDF).
L'uso della modellistica matematica può essere d'aiuto nella scelta della terapia dialitica più adatta, e il presente lavoro di tesi ha come obiettivo quello di fornire un tale strumento per il caso particolare dell'\textit{on-line} HDF. 

Il modello qui proposto è basato sulla descrizione della cinetica di alcuni soluti presenti nel sangue che si ritiene importante monitorare durante la dialisi (sodio, potassio, cloro, calcio, fosfato inorganico, magnesio, urea, creatinina, proteine), sulla descrizione degli scambi di fluidi tra i diversi compartimenti e delle variazioni pressorie, durante una seduta dialitica effettuata con la tecnica dell'emodiafiltrazione. In particolare, si è cercato di sintetizzare matematicamente in un unico sistema di equazioni differenziali ordinarie, le interazione che avvengono tra i seguenti tre sistemi:
\begin{enumerate}
	\item il sistema ``paziente'', descritto come composto da tre compartimenti (plasmatico, interstiziale e intracellulare) separati fra loro da membrane semipermeabili;
	\item il sistema di produzione ``on-line'' del fluido dializzante e del fluido sostitutivo del plasma\footnote{in questa tesi ci riferiremo a questo fluido come fluido di sostituzione o fluido di diluizione, per distinguerlo dal fluido dializzante.};
	\item il filtro dializzatore, sede degli scambi di massa e volume fra plasma e liquido di dialisi.
\end{enumerate}
Nel sistema ``paziente'', lo scambio di massa è bicompartimentale e avviene fra compartimento intracellulare e compartimento extracellulare, quest'ultimo formato dal compartimento plasmatico e interstiziale. Lo scambio di fluidi è di tipo tri-compartimentale (plasma, interstizio, cellule).

Il modello è stato implementato utilizzando il software per il calcolo numerico Matlab\textsuperscript\textregistered{} 7 (R14), e richiede l'immissione dei dati iniziali del paziente e delle impostazioni della macchina dializzatrice. Nel modello sono stati inseriti e analizzati in particolare tre parametri: il coefficiente di trasferimento di massa alla membrana cellulare $k$, il parametro adimensionale $\rho$, che tiene conto della variabilità della permeabilià dei capillari, e il coefficiente $\eta$ , che rappresenta la capacità di estrazione del dializzatore, influenzata dallo strato proteico che si forma lungo i capillari del filtro nel corso della dialisi. Per questi parametri è stata effettuata un'analisi di sensitività al fine di poter comprendere quali tra essi influenzano in maniera preponderante le uscite del modello; i valori iniziali di questi parametri sono stati fatti variare singolarmente del $\pm 10\%$ mantenendo costanti tutti gli altri valori e calcolando così gli scostamenti rispetto al valore di riferimento. Quello che si è dedotto dall'analisi è che il parametro relativo ad un determinato soluto influenza sia, come era prevedibile, la dinamica del soluto stesso, ma anche quella degli altri soluti, evidenziando così il fatto che il modello è stato sviluppato con i soluti che convivono e interagiscono in un unico ambiente.

Uno degli obiettivi finali è stato quello di valutare per quanto tempo i parametri di una seduta potessero risultare validi per la predizione delle sedute successive, in modo da poter fornire al personale medico un utile strumento di predizione. Il modello qui esposto mostra nel complesso risultati conformi e vicini a quelli ottenuti tramite rilevazioni cliniche e può quindi essere ritenuto un valido strumento di simulazione. Gli scostamenti relativi alle concentrazioni plasmatiche dei soluti di interesse risultano infatti quasi tutti inferiori al $20\%$. In particolare i risultati migliori sono stati ottenuti per sodio, potassio, cloro, calcio e magnesio, con scostamenti sempre inferiori al $10\%$. Considerando invece i valori medi degli scostamenti, le percentuali d'errore scendono per tutti al di sotto del $10\%$. In relazione ai risultati ottenuti vengono fornite analisi dettagliate per ogni soluto, affinchè possano essere apportate al modello modifiche e accorgimenti che ne aumentino la capacità descrittiva.

Infine, si è mostrato come sia possibile utilizzare il modello per individuare delle differenze fra le tecniche di diluizione durante on-line HDF (pre, post), valutandone i diversi effetti sulle dinamiche di scambio dei soluti nel corso della dialisi. Si sono individuati il grado di compartimentazione e la carica elettrica dei soluti come fattori che determinano la capacità estrattiva dell'una o dell'altra tecnica di diluizione tra quelle disponibili per l'on-line HDF.
