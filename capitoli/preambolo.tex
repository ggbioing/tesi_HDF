\documentclass[10pt,twoside]{book}
\usepackage{amssymb,amsmath}         % pacchetto per la gestione delle formule matematiche
\usepackage[english, italian]{babel} % regole di sillabazione
\usepackage[T1]{fontenc}
\usepackage[utf8]{inputenc}
\usepackage[dvips]{epsfig,psfrag}    % pacchetto per la gestione delle figure *.eps
\usepackage{a4}                      % formato di stampa A4
\usepackage{cancel}
\usepackage{booktabs}
\usepackage{gensymb}
\usepackage{tabularx}

\usepackage{fancyhdr}
%\headsep = 30pt
\pagestyle{fancy} 
% I comandi seguenti impediscono la scrittura in maiuscolo 
% dei nomi dei capitoli e dei paragrafi nelle testatine 
\renewcommand{\chaptermark}[1]{\markboth{#1}{}} 
\renewcommand{\sectionmark}[1]{\markright{\thesection\ #1}} 
\fancyhf{} % Rimuove l'attuale contenuto della testatina 
% e del piè di pagina 
\fancyhead[LE,RO]{\thepage} 
\fancyhead[LO]{\rightmark} 
\fancyhead[RE]{\leftmark} 
\renewcommand{\headrulewidth}{0.5pt} 
\renewcommand{\footrulewidth}{0pt} 
\addtolength{\headheight}{0.5pt} % Riserva spazio per la linea 
\fancypagestyle{plain}{% 
\fancyhead{} % Ignora, nello stile plain, le testatine e la linea 
\renewcommand{\headrulewidth}{0pt}} 

\usepackage{setspace}
\usepackage{graphicx}
\usepackage{subfigure}
\usepackage[hang,small,bf]{caption}
\setlength{\captionmargin}{20pt}
\usepackage{a4}
\usepackage{natbib}
%\usepackage{indentfirst}
\usepackage{wrapfig}

% Carico il pacchetto color e 
% definisco i colori uguali a quelli di default di matlab
\usepackage{color}
\definecolor{commenti}{rgb}{0.13,0.55,0.13}
\definecolor{stringhe}{rgb}{0.63,0.125,0.94}

\usepackage{listings} % inserisce listati di programmi
\lstloadlanguages{Matlab}
\lstset{
basicstyle=\small\ttfamily,
basicstyle = \scriptsize, % print whole listing small
keywordstyle = \color{blue},% blue keywords
identifierstyle =, % nothing happens
commentstyle = \color{commenti}, % comments
stringstyle = \ttfamily \color{stringhe}, % typewriter type for strings
showstringspaces = false, % no special string spaces
emph = {for, if, then, else, end},
emphstyle = \color{blue},
firstnumber = 1, % numero della prima linea
numbers = left, %  show number_line
numberstyle = \tiny, % style of number_line
stepnumber = 5, % one number_line after stepnumber
numbersep = 5pt,
language = {Matlab}, % per riconoscere la sintassi matlab
extendedchars = true, % per abilitare caratteri particolari
breaklines = true, % per mandare a capo le righe troppo lunghe
breakautoindent = true, % indenta le righe spezzate
breakindent = 30pt, % indenta le righe di 30pt
frame=lines,
deletekeywords = {beta, gamma},
morekeywords={fmincon},
literate=
{à}{{\ttfamily\`a}}1
{è}{{\ttfamily\`e}}1
{ì}{{\ttfamily\`i}}1
{ò}{{\ttfamily\`o}}1
{ù}{{\ttfamily\`u}}1,
}
